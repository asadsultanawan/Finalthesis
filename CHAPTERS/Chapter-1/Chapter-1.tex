\chapter{Introduction}
\label{1}
\section{Motivation}

Rising traffic in the country gives birth to rising accidents. To 
avoid/minimize the accidents, proper surveillance is required. Detection 
and tracking of vehicles have become a key component in traffic 
surveillance and automatic driving. Traditional algorithms like Gaussian 
Mixed Model (GMM) have achieved encouraging success in this field \cite{chap_1_article:1} but due 
to variation in illumination, occlusion, background clutter etc. detection 
of vehicles is still a challenge. 
Deep Neural Networks (DNNs) have gained a lot of attention in 
past few years. Due to the advancement  in deep learning research, object 
detection has achieved major advancements in recent years. The work 
in this project is based on DNNs to detect and classify vehicles from 
a dataset collected from images and videos.

\section{Background}

Classical methods of vehicle detection were based on some 
sort of features like symmetry, edges, texture, underbody shadows 
and corners \cite{chap_1_article:2}. These methods were computationally less demanding and
worked well with certain environmental settings. However, these methods fail in many situations 
in scenarios like less illuminated highways. With the advent of 
deep learning which is a subset of machine learning, object detection and 
object classification have dramatically improved. Machine learning 
and deep learning are sub-branches of artificial intelligence which  
have been applied to fields including computer vision, machine vision,
natural language processing, speech processing, biomedical imaging,
and traffic surveillance etc. 

In our research we are going to use deep learning techniques for detection
and classification of different kind of vehicles on the road.
We are going to use deep convolutional neural networks (DCNNs)
based architectures for detection,localization and classification of
different types of vehicles in still images.  

\section{Goals \& Objectives}

The goal of this project is to make a deep learning based model (using Python) for vehicle detection and classification. Main objectives are summarized as:
\begin{itemize}
\item Collection of datasets (images dataset of car, truck classes etc. and videos).
\item Design of Convolutional Neural Network
\item Configuration of training options (AlexNet, VGGNet, ResNet etc.)
\item Training of detector (RCNN, Fast- RCNN, Faster-RCNN, YOLO etc.)
\item Evaluation of the trained detector
\end{itemize}

\section{Challenges}

One of the big challenges that deep learning based methods
face is the availability of quality data. These methods work
best when large amounts of quality data is used. However,
when sufficient quality data is not fed in a deep learning system,
it can fail quite badly. In our case, gathering sufficient, high-quality,
context-relevant data can be a challenge.
Furthermore, deep learning architectures are essentially bulky,
compute intensive algorithms and in order to implement these algorithms
we need access to high-end computing machines specifically designed
for this purpose. Typically, high-end GPUs are used for implementing
these algorithms. These high-end computing machines can be
remotely accessed in the cloud however, duration for which these
machines are available and affordability can be a challenge in our case.	

\section{Organization of Thesis}

\textbf{Chapter \ref{Chapter 2}}: Chapter 2 is about literature review. It is about
the whole reading and understanding of image classification \& detection.

\noindent\textbf{Chapter \ref{Chapter 3}}: This chapter is about Convolutional Neural Networks. All CNN architectures with their details
are discussed.

\noindent\textbf{Chapter \ref{Chapter 4}}: This chapter contains all
the procedural details of the dataset formation for classifier,
training process and evaluation of results along with all the relevant codes.

\noindent\textbf{Chapter \ref{Chapter 5}}: This chapter deals with\
the training and testing of a YOLO based cutom object detector.
All the relavant codes and results are attached for easy understanding.
